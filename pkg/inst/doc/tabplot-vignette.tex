%\VignetteIndexEntry{tabplot}
\documentclass[11pt, fleqn, a4paper]{article}
\usepackage[english]{babel}
\usepackage{amsmath, amssymb}
\usepackage{natbib}
\usepackage{algpseudocode}
\usepackage{algorithm}
\renewcommand{\algorithmicrequire}{\textbf{Input:}}
\renewcommand{\algorithmicensure}{\textbf{Output:}}
\usepackage{threeparttable}

% stimulate latex to put multiple floats on a page.
\setcounter{topnumber}{2}
\setcounter{bottomnumber}{2}
\setcounter{totalnumber}{3}
\setcounter{dbltopnumber}{2}
\renewcommand{\topfraction}{.9}
\renewcommand{\textfraction}{.1}
\renewcommand{\bottomfraction}{.75}
\renewcommand{\floatpagefraction}{.9}
\renewcommand{\dblfloatpagefraction}{.9}
\renewcommand{\dbltopfraction}{.9}
\hyphenation{time-stamp}

\title{Correction of rounding, typing, and sign errors with the {\tt tabplot} package}
\author{Martijn Tennekes and Edwin de Jonge}
\usepackage{Sweave}
\begin{document}
\maketitle
\begin{abstract}

The tableplot is a powerful visualization method to explore and analyse large multivariate datasets. In this vignette, the implementation of tableplots in R is described. 


\end{abstract}

\maketitle

\newpage

\tableofcontents
\listofalgorithms
\newpage
\section{Introduction}
We will use the diamonds dataset that is provided in the {\tt ggplot2} package. This dataset contains information about 53,940 diamonds. There are 7 continuous variables and 3 categorical. In order to illustrate the visualization of missing values, we add several NA's.

\begin{Schunk}
\begin{Sinput}
> require(ggplot2)
> data(diamonds)
> is.na(diamonds$price) <- diamonds$cut == "Ideal"
> is.na(diamonds$cut) <- (runif(nrow(diamonds)) > 0.8)
> diamonds$expensive <- diamonds$price >= 10000
\end{Sinput}
\end{Schunk}

A tableplot is simply created by the function {\tt tableplot}.
\begin{Schunk}
\begin{Sinput}
> tableplot(diamonds)
\end{Sinput}
\end{Schunk}

By the other arguments we can customize the tableplot. The most important arguments to start with are: {\tt colNames}, a vector of the column names that are plotted and {\tt sortCol}, a vector of column names or indices on which the data is ordered.
\begin{Schunk}
\begin{Sinput}
> tableplot(diamonds, colNames = c("carat", "price", "cut", "color", 
+     "clarity"), sortCol = "price")
\end{Sinput}
\end{Schunk}



\section{Final remarks}
Conclusions


%\bibliographystyle{chicago}
%\bibliography{deducorrect}

\newpage
\appendix
\section{First appendix on the {\tt tabplot} package}
\section{Second appendix}




\end{document}
